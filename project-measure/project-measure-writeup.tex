\documentclass[]{article}
\usepackage[left=1in,top=1in,right=1in,bottom=1in]{geometry}


%%%% more monte %%%%
% thispagestyle{empty}
% https://stackoverflow.com/questions/2166557/how-to-hide-the-page-number-in-latex-on-first-page-of-a-chapter
\usepackage{color}
% \usepackage[table]{xcolor} % are they using color?

% \definecolor{WSU.crimson}{HTML}{981e32}
% \definecolor{WSU.gray}{HTML}{5e6a71}

% \definecolor{shadecolor}{RGB}{248,248,248}
\definecolor{WSU.crimson}{RGB}{152,30,50} % use http://colors.mshaffer.com to convert from 981e32
\definecolor{WSU.gray}{RGB}{94,106,113}

%%%%%%%%%%%%%%%%%%%%%%%%%%%%

\newcommand*{\authorfont}{\fontfamily{phv}\selectfont}
\usepackage{lmodern}


  \usepackage[T1]{fontenc}
  \usepackage[utf8]{inputenc}




\usepackage{abstract}
\renewcommand{\abstractname}{}    % clear the title
\renewcommand{\absnamepos}{empty} % originally center

\renewenvironment{abstract}
 {{%
    \setlength{\leftmargin}{0mm}
    \setlength{\rightmargin}{\leftmargin}%
  }%
  \relax}
 {\endlist}

\makeatletter
\def\@maketitle{%
  \pagestyle{empty}
  \newpage
%  \null
%  \vskip 2em%
%  \begin{center}%
  \let \footnote \thanks
    {\fontsize{18}{20}\selectfont\raggedright  \setlength{\parindent}{0pt} \@title \par}%
}
%\fi
\makeatother






\usepackage{color}
\usepackage{fancyvrb}
\newcommand{\VerbBar}{|}
\newcommand{\VERB}{\Verb[commandchars=\\\{\}]}
\DefineVerbatimEnvironment{Highlighting}{Verbatim}{commandchars=\\\{\}}
% Add ',fontsize=\small' for more characters per line
\usepackage{framed}
\definecolor{shadecolor}{RGB}{248,248,248}
\newenvironment{Shaded}{\begin{snugshade}}{\end{snugshade}}
\newcommand{\AlertTok}[1]{\textcolor[rgb]{0.94,0.16,0.16}{#1}}
\newcommand{\AnnotationTok}[1]{\textcolor[rgb]{0.56,0.35,0.01}{\textbf{\textit{#1}}}}
\newcommand{\AttributeTok}[1]{\textcolor[rgb]{0.77,0.63,0.00}{#1}}
\newcommand{\BaseNTok}[1]{\textcolor[rgb]{0.00,0.00,0.81}{#1}}
\newcommand{\BuiltInTok}[1]{#1}
\newcommand{\CharTok}[1]{\textcolor[rgb]{0.31,0.60,0.02}{#1}}
\newcommand{\CommentTok}[1]{\textcolor[rgb]{0.56,0.35,0.01}{\textit{#1}}}
\newcommand{\CommentVarTok}[1]{\textcolor[rgb]{0.56,0.35,0.01}{\textbf{\textit{#1}}}}
\newcommand{\ConstantTok}[1]{\textcolor[rgb]{0.00,0.00,0.00}{#1}}
\newcommand{\ControlFlowTok}[1]{\textcolor[rgb]{0.13,0.29,0.53}{\textbf{#1}}}
\newcommand{\DataTypeTok}[1]{\textcolor[rgb]{0.13,0.29,0.53}{#1}}
\newcommand{\DecValTok}[1]{\textcolor[rgb]{0.00,0.00,0.81}{#1}}
\newcommand{\DocumentationTok}[1]{\textcolor[rgb]{0.56,0.35,0.01}{\textbf{\textit{#1}}}}
\newcommand{\ErrorTok}[1]{\textcolor[rgb]{0.64,0.00,0.00}{\textbf{#1}}}
\newcommand{\ExtensionTok}[1]{#1}
\newcommand{\FloatTok}[1]{\textcolor[rgb]{0.00,0.00,0.81}{#1}}
\newcommand{\FunctionTok}[1]{\textcolor[rgb]{0.00,0.00,0.00}{#1}}
\newcommand{\ImportTok}[1]{#1}
\newcommand{\InformationTok}[1]{\textcolor[rgb]{0.56,0.35,0.01}{\textbf{\textit{#1}}}}
\newcommand{\KeywordTok}[1]{\textcolor[rgb]{0.13,0.29,0.53}{\textbf{#1}}}
\newcommand{\NormalTok}[1]{#1}
\newcommand{\OperatorTok}[1]{\textcolor[rgb]{0.81,0.36,0.00}{\textbf{#1}}}
\newcommand{\OtherTok}[1]{\textcolor[rgb]{0.56,0.35,0.01}{#1}}
\newcommand{\PreprocessorTok}[1]{\textcolor[rgb]{0.56,0.35,0.01}{\textit{#1}}}
\newcommand{\RegionMarkerTok}[1]{#1}
\newcommand{\SpecialCharTok}[1]{\textcolor[rgb]{0.00,0.00,0.00}{#1}}
\newcommand{\SpecialStringTok}[1]{\textcolor[rgb]{0.31,0.60,0.02}{#1}}
\newcommand{\StringTok}[1]{\textcolor[rgb]{0.31,0.60,0.02}{#1}}
\newcommand{\VariableTok}[1]{\textcolor[rgb]{0.00,0.00,0.00}{#1}}
\newcommand{\VerbatimStringTok}[1]{\textcolor[rgb]{0.31,0.60,0.02}{#1}}
\newcommand{\WarningTok}[1]{\textcolor[rgb]{0.56,0.35,0.01}{\textbf{\textit{#1}}}}
\usepackage{longtable,booktabs}



\title{\textbf{\textcolor{WSU.crimson}{A Study of Human Proportions,
Climbing Advantage, and Limb
Dominance}} \newline \textbf{\textcolor{WSU.gray}{Insights from the FALL
2020 STAT 419 Measure Dataset}}  }
 

%  

% \author{ \Large true \hfill \normalsize \emph{} }
\author{\Large Lindsey
Kornowske\vspace{0.05in} \newline\normalsize\emph{Washington State
University}  }


\date{November 06, 2020}
\setcounter{secnumdepth}{3}

\usepackage{titlesec}
% See the link above: KOMA classes are not compatible with titlesec any more. Sorry.
% https://github.com/jbezos/titlesec/issues/11
\titleformat*{\section}{\bfseries}
\titleformat*{\subsection}{\bfseries\itshape}
\titleformat*{\subsubsection}{\itshape}
\titleformat*{\paragraph}{\itshape}
\titleformat*{\subparagraph}{\itshape}

% https://code.usgs.gov/usgs/norock/irvine_k/ip-092225/


%\titleformat*{\section}{\normalsize\bfseries}
%\titleformat*{\subsection}{\normalsize\itshape}
%\titleformat*{\subsubsection}{\normalsize\itshape}
%\titleformat*{\paragraph}{\normalsize\itshape}
%\titleformat*{\subparagraph}{\normalsize\itshape}

% https://tex.stackexchange.com/questions/233866/one-column-multicol-environment#233904
\usepackage{environ}
\NewEnviron{auxmulticols}[1]{%
  \ifnum#1<2\relax% Fewer than 2 columns
    %\vspace{-\baselineskip}% Possible vertical correction
    \BODY
  \else% More than 1 column
    \begin{multicols}{#1}
      \BODY
    \end{multicols}%
  \fi
}





\usepackage{natbib}
\setcitestyle{aysep={}} %% no year, comma just year
% \usepackage[numbers]{natbib}
\bibliographystyle{./../biblio/ormsv080.bst}



\usepackage[strings]{underscore} % protect underscores in most circumstances




\newtheorem{hypothesis}{Hypothesis}
\usepackage{setspace}


%%%%%%%%%%%%%%%%%%%%%%%%%%%%%%%%%%%%%%%%%%%%%%%%%%%%%
%%% MONTE ADDS %%%

\usepackage{fancyhdr} % fancy header 
\usepackage{lastpage} % last page 

\usepackage{multicol}


\usepackage{etoolbox}
\AtBeginEnvironment{quote}{\singlespacing\small}
% https://tex.stackexchange.com/questions/325695/how-to-style-blockquote


\usepackage{soul}			%% allows strike-through
\usepackage{url}			%% fixes underscores in urls
\usepackage{csquotes}		%% allows \textquote in references
\usepackage{rotating}		%% allows table and box rotation
\usepackage{caption}		%% customize caption information
\usepackage{booktabs}		%% enhance table/tabular environment
\usepackage{tabularx}		%% width attributes updates tabular
\usepackage{enumerate}		%% special item environment
\usepackage{enumitem}		%% special item environment

\usepackage{lineno}		%% allows linenumbers for editing using \linenumbers
\usepackage{hanging}


\usepackage{mathtools}  	%% also loads amsmath
\usepackage{bm}		%% bold-math
\usepackage{scalerel}	%% scale one element (make one beta bigger font)

\newcommand{\gFrac}[2]{ \genfrac{}{}{0pt}{1}{{#1}}{#2} }

\newcommand{\betaSH}[3]{  \gFrac{\text{\tiny #1}}{{\text{\tiny #2}}}\hat{\beta}_{\text{#3}}   }
\newcommand{\betaSB}[3]{              ^{\text{#1}} _{\text{#2}} \bm{\beta} _{\text{#3}}                   }  %% bold
\newcommand{\bigEQ}{  \scaleobj{1.5}{{\ }= } }
\newcommand{\bigP}[1]{  \scaleobj{1.5}{#1 } }





\usepackage{endnotes}  % he already does this ...
\renewcommand{\enotesize}{\normalsize}
% https://tex.stackexchange.com/questions/99984/endnotes-do-not-be-superscript-and-add-a-space
\renewcommand\makeenmark{\textsuperscript{[\theenmark]}} % in brackets %
% https://tex.stackexchange.com/questions/31574/how-to-control-the-indent-in-endnotes
\patchcmd{\enoteformat}{1.8em}{0pt}{}{}

\patchcmd{\theendnotes}
  {\makeatletter}
  {\makeatletter\renewcommand\makeenmark{\textbf{[\theenmark]} }}
  {}{}



% https://tex.stackexchange.com/questions/141906/configuring-footnote-position-and-spacing

\addtolength{\footnotesep}{5mm} % change to 1mm

\renewcommand{\thefootnote}{\textbf{\arabic{footnote}}}
\let\footnote=\endnote
%\renewcommand*{\theendnote}{\alph{endnote}}
%\renewcommand{\theendnote}{\textbf{\arabic{endnote}}}


\renewcommand*{\notesname}{NOTES}

\makeatletter
\def\enoteheading{\section*{\notesname
  \@mkboth{\MakeUppercase{\notesname}}{\MakeUppercase{\notesname}}}%
  \mbox{}\par\vskip-2.3\baselineskip\noindent\rule{.5\textwidth}{0.4pt}\par\vskip\baselineskip}
\makeatother


\renewcommand*{\contentsname}{CONTENTS}

\renewcommand*{\refname}{REFERENCES}


%\usepackage{subfigure}
\usepackage{subcaption}

\captionsetup{labelfont=bf}  % Make Table / Figure bold

%%% you could add elements here ... monte says .... %%%
%\usepackage{mypackageForCapitalH}


%%%%%%%%%%%%%%%%%%%%%%%%%%%%%%%%%%%%%%%%%%%%%%%%%%%%%

% set default figure placement to htbp
\makeatletter
\def\fps@figure{htbp}
\makeatother

\usepackage{booktabs}
\usepackage{longtable}
\usepackage{array}
\usepackage{multirow}
\usepackage{wrapfig}
\usepackage{float}
\usepackage{colortbl}
\usepackage{pdflscape}
\usepackage{tabu}
\usepackage{threeparttable}
\usepackage{threeparttablex}
\usepackage[normalem]{ulem}
\usepackage{makecell}

% move the hyperref stuff down here, after header-includes, to allow for - \usepackage{hyperref}

\makeatletter
\@ifpackageloaded{hyperref}{}{%
\ifxetex
  \PassOptionsToPackage{hyphens}{url}\usepackage[setpagesize=false, % page size defined by xetex
              unicode=false, % unicode breaks when used with xetex
              xetex]{hyperref}
\else
  \PassOptionsToPackage{hyphens}{url}\usepackage[draft,unicode=true]{hyperref}
\fi
}

\@ifpackageloaded{color}{
    \PassOptionsToPackage{usenames,dvipsnames}{color}
}{%
    \usepackage[usenames,dvipsnames]{color}
}
\makeatother
\hypersetup{breaklinks=true,
            bookmarks=true,
            pdfauthor={Lindsey Kornowske (Washington State University)},
             pdfkeywords = {vitruvian man; biometric; ape-index;
correlation; WSU STATS 419},  
            pdftitle={A Study of Human Proportions, Climbing Advantage,
and Limb Dominance: Insights from the FALL 2020 STAT 419 Measure
Dataset},
            colorlinks=true,
            citecolor=blue,
            urlcolor=blue,
            linkcolor=magenta,
            pdfborder={0 0 0}}
\urlstyle{same}  % don't use monospace font for urls

% Add an option for endnotes. -----

%
% add tightlist ----------
\providecommand{\tightlist}{%
\setlength{\itemsep}{0pt}\setlength{\parskip}{0pt}}

% add some other packages ----------

% \usepackage{multicol}
% This should regulate where figures float
% See: https://tex.stackexchange.com/questions/2275/keeping-tables-figures-close-to-where-they-are-mentioned
\usepackage[section]{placeins}



\pagestyle{fancy}   
\lhead{\textcolor{WSU.crimson}{\textbf{ A Study of Human Proportions,
Climbing Advantage, and Limb Dominance }}}
\chead{}
\rhead{\textcolor{WSU.gray}{\textbf{  Page\ \thepage\ of\ \protect\pageref{LastPage} }}}
\lfoot{}
\cfoot{}
\rfoot{}


\begin{document}
	
% \pagenumbering{arabic}% resets `page` counter to 1 
%    

% \maketitle

{% \usefont{T1}{pnc}{m}{n}
\setlength{\parindent}{0pt}
\thispagestyle{plain}
{\fontsize{18}{20}\selectfont\raggedright 
\maketitle  % title \par  

}

{
   \vskip 13.5pt\relax \normalsize\fontsize{11}{12} 
   
\textbf{\authorfont Lindsey
Kornowske} \hskip 15pt \emph{\small Washington State University}   

}

}








\begin{abstract}

    \hbox{\vrule height .2pt width 39.14pc}

    \vskip 8.5pt % \small 

\noindent This study uses a student-collected dataset containing
quantitative biometric variables inspired by the Leonardo Da Vinci's
Vitruvian Man to investigate how these measurements are correlated with
height, how relative arm span to height size compares to that of famous
rock climbers, and how limb dominance may relate to size assymetries. A
combination of correlation analysis, summary statistics, and kmeans were
used to evaluate these questions. Height was found to be the best
predictor for hand-elbow length, foot length, floor to hip height, and
floor to armpit height, and not a good predictor for hand length. The
average ape-index for the climbers (2.3 inches) was greater than that of
the class dataset (0 inches), but the standard deviation was much lower.
More members of the class dataset had dominant limbs that were smaller
than their nondominant limbs. \vspace{0.25in}


\vskip 8.5pt \noindent \textbf{\underline{Keywords}:} vitruvian man;
biometric; ape-index; correlation; WSU STATS 419 \par

    




    
    \hbox{\vrule height .2pt width 39.14pc}
    \vskip 5pt 
    \hfill \textbf{\textcolor{WSU.gray}{ November 06, 2020 } }
    \vskip 5pt 
    
\end{abstract}


\vskip -8.5pt



 % removetitleabstract

\noindent  

\section{Introduction}
\label{sec:intro}

\indent The Fall 2020 STAT 419 measure dataset presents an interesting
opportunity to investigate which biometrics are best associated with
height. According to Leonardo Da Vinci's Vitruvian Man, the
perfectly-proportioned man is a paragon of mathematics: his body is
eight times his head, four times his shoulder width, knee height, and
thigh length, and two times his groin height. And yet, a modern study of
65 thousand Air Force recruits found that these idealized proportions
were not represented true to Vitruvian standards \citep{Thomas:2020}. If
all persons in the dataset assessed herein were ``ideal'' by Vitruvian
standards, all correlations between height and other biometrics would be
1, therefore it is far more interesting to understand the relationship
of these variables as they exist in true populations. I can best
illustrate this bias with a personal example: I am as tall as my mother,
but my feet are 2 shoe sizes smaller, possibly because I danced pointe
as a child before my bones fully developed. Given the variability that
may be introduced by gender, ethnicity, age, and lifestyle factors, are
there biometrics that are well-correlated with height? This evaluation
will use correlation to evaluate where these stronger relationships lie.

\indent A secondary question of interest related to this dataset was
inspired by my experience in the world of sport climbing, where it is
fashionable to conflate climbing ability to ape index. The ape-index,
which may be calculated either as the wingspan minus height, or the
wingspan divided by height, enumerates an individual's reach relative to
their height. In a sport where the difference between victory and a
critical fall may be a game of a few centimeters, it is easy to
understand why sponsors proudly display photos of their athletes
immediately aside this ``all-important'' metric. For the record, Da
Vinci's man, with an arm span 4/5 the length of total body height, would
have made an exceptionally poor climber by this standard
\citep{Thomas:2020}. As an ammature rock climber, I have an anecdotal
understanding that the most elite athletes in my community tend to have
a positive ape index. Researchers have evaluated this phenomenon in
assorted climbing groups. \citet{Watts:2003} found that a set of
competitive child climbers (average age of 13) had a significantly
greater (p = 0,02) ape index (cm/cm) of 1.01, relative to non-climbers
of similar age. Within groups of climbers, however, Ape Index was not
significantly different between advanced and elite athletes
\citep{Ozimek:2017}. Given this, I wanted to understand how Ape Index
data for famous climbers (for more information on data collection see
section \ref{sec:climbdat}) compared with our Fall 2020 STAT 419 measure
dataset.

\indent The STAT 419 measure dataset also contains qualitative
variables. I was interested in understanding how two variables that
would pertain to almost everyone in the dataset - writing hand and
swinging arm dominance - related to the measures of size in both the
hand length and arm length, respectively. In a study of 196
participants, \citet{Maleki:2019} found that hand dominance was
associated with 5.3\% to 7.5\% greater strength relative to the
non-dominant hand. Furthermore, longer hand and forearm length were
positively correlated with greater pinch strength \citep{Maleki:2019}.
Based on this, my hypothesis is that the dominant hand will be larger
than the non-dominant hand in the dataset, and that this pattern will
also be observed in the relationship between dominant swinging arm and
larger arm.

\begin{figure}[!ht]
    \hrule
    \caption{ \textbf{The New Ideal?} }
    \begin{center}
        \scalebox{1.00}{    \includegraphics[trim = 0 0 0 0,clip,width=0.85\textwidth]{figures/newideal.jpg} }
    \end{center}
    \label{fig:handout-1}
    \hrule
\end{figure}

Image modified from \citet{Thomas:2020}. Da Vinci's Vitruvian Man
reflected concepts of ideal symmetry and proportion, yet the meaning of
ideal is fluid depending on an individual's lifestyle. This study
investigates the Vitruvian proportions as they pertain to the STAT 419
student measure dataset, how one's status as a professional sport
climber makes them more or less likely to exceed average wingspan
proportions, and how dominance is associated with relative side size.

\section{Research Question:  Which biometrics are best correlated with height?}
\label{sec:rq}

This question will be investigated using correlations of select
biometrics scaled relative to the height. Then the means will be used to
determine the STAT 419 student measure dataset values for the
proportions listed by \citet{Thomas:2020}.

\subsection{How do famous climbers compare to the FALL 2020 STAT 419 Measure dataset for ape index?} 
\label{sec:rq2}

A supplementary dataset containing the Ape-Index of 29 professional
climbers will be collected for comparison to the STAT 419 measure
dataset described in section. The average ape-index will be calculated
for each dataset. Then Kmeans clustering will be applied on the combined
datasets to determine how the climbers group among the class data.

\subsection{Are dominant limbs of consistent greater or lesser relative size?}
\label{sec:rq3}

Count information will be collected for each observance of dominant side
being greater than or lesser than the non-dominant side. Ambidextrous
individuals and persons missing data on either side will not be counted.

\section{Data Description}
\label{sec:data}

The Fall 2020 Stat 419 measure dataset was collected by the enrolled
students in August and September of 2020. The data contains 23
quantitative biometric variables and 9 supplementary variables for each
student, and approximately 9 members of their family and friends. The
final dataset compiled 428 observations that may have been collected by
the students themselves, or by participants due to COVID-19
social-distancing guidelines in place at the time of data collection.
For this reason, data collection practices were not uniform; some
observations are missing, duplicated, and in some cases both left and
right side information is provided.

\subsubsection{Climbers Measure Data}
\label{sec:climbdat}

The climbers measure dataset is comprised of name, height, wingspan,
ape-index, gender, and source data for 29 professional rock climbers
\citep{99Boulders, youtube, Sportiva, GymClimber2019a}, Ape index was
calculated as in the STAT 419 measure dataset by subtracting the height
from wingspan.

\subsection{Summary Statistics of Data}
\label{sec:data-summary}

Summary statistics and correlation tables of the STAT 419 measure data
can be found in Section \ref{sec:anrq}.

\section{Key Findings}
\label{sec:findings}

\subsubsection{Findings: Which biometrics are best correlated with height?}
\label{sec:findRQ}

All biometric measurements had significant correlations relative to
total height. Hand-elbow length and foot-length were as strongly
correlated with height (0.95), as floor to hip and floor to armpit. This
conflicts somewhat with my hypothesis that lifestyle factors may
contribute to relative foot size, and suggests that overall, height is a
good predictor for footlength. Hand-length was the least correlated with
height (0.38), which was unexpected given that the hand-to-elbow
measurement also includes the length of the hand.

Another observation of interest was that foot-length was very highly
correlated with hand-to-elbow (0.97), The play-ground myth that one's
foot is exactly the size of their forearm may have some merit given this
finding.

The averages collected as a part of this first research question were
used to calculate the proportions of interest that were also calculated
in the paper by \citet{Thomas:2020}.

Proportions to body length: Head Height: 0.13 Arm span: 1.00 Groin
height: 0.57 Knee height: 0.27 Thigh length: 0.30

Head height and knee height proportions were very close to the Vitruvian
man proportions of 0.125 and 0.25, respectively. The arm span proportion
was the most different; much greater at 1, compared with 0.8. Groin
height was also larger - 0.57 compared to 0.5 - but I used the floor to
hip measurement, which likely leads to an over-approximation of this
measurement. This over-approximation may also be responsible for the
fact that the thigh length proportion was greater - 0.3 compared with
0.25 - as thigh length was calculated by subtracting the knee height
from the hip-height value.

\subsubsection{Findings: How do famous climbers compare to the FALL 2020 STAT 419 Measure dataset for ape index?}
\label{sec:findRQ2}

The average wingspan of the climbers dataset is 2.2 inches with a
standard deviation of 1.76 inches. This is much greater than the average
of the student measure dataset, 0.0 inches; the standard deviation of
the student measure dataset is 7.95 inches, which suggests that while
the subset of climbers averaged greater, more extremes are present in
the student dataset.

\subsubsection{Findings: Are dominant limbs of consistent greater or lesser relative size?}
\label{sec:findRQ3}

Dominance appears to be associated with the smaller limb of the two
sides. In 51\% of the observations of the sample subset, the dominant
swinging arm was smaller than the non-dominant arm. In 52\% of the
observations of the sample subset, the dominant writing hand was smaller
than the non-dominant hand.

\section{Conclusion}
\label{sec:conclusion}

\indent The FALL 2020 STAT 419 Student measure dataset was collected at
the beginning of the Fall 2020 semester by student participants and
compiled for this analysis. Upon appropriate cleaning of the data, three
questions were formulated, informed by the study written by
\citet{Thomas:2020} examining the proportions behind Da Vinci's
Vitruvian Man with a modern dataset.

\indent The class data was comparable to the Vitruvian man for the
proportions of head height and knee height, but were greater for the
proportions of armspan, thigh length, and groin height. Height was found
to be the best predictor for hand-elbow length, foot length, floor to
hip height, and floor to armpit height, and not a good predictor for
hand length.

\indent A dataset of height, wingspan, and ape-index measurements was
collected for 29 professional climbers. The average of this dataset was
greater than that of the class dataset, as was somewhat expected based
on work conducted by \citet{Ozimek:2017}, although it is not clear how
conclusive this comparison is, given the high standard deviation for
ape-index that was observed in the class dataset. Kmeans clustering of
the results did not suggest any particular trends for the climbers among
the class data. This is likely a product of the fact that some of the
data collectors in the STAT 419 may have misunderstood armspan
measurement. At least one data collector regularly produced ape-index
values of above 15, which if biologically possible, is expected to be
extremely unlikely. Further analysis of this data should involve removal
of such outliers, although data removal is a process that can easily
contribute to skewed results. This criteria should not be taken lightly,
and was considered to be beyond the scope of this work at this time.

\indent A preliminary look into the assocations between dominance and
size was completed based on count data. These data show that more
persons in the STAT 419 dataset had dominant limbs that were smaller
than their nondominant limbs. However, statistics were not applied to
this assessment and to conclude based on count data is to accept a high
Type II error risk. However, this finding was interesting in that it
conflicted with a study that was found assessing the relationship
between dominance, size, and grip strength, which lead to the hypothesis
that the dominant limb would more often be of greater size
\citep{Maleki:2019}. However, the scope of the size difference was not
evaluated as it was in this study, it is possible that these differences
are rather nominal and do not have a large impact.

\indent In conclusion, these data were sufficient for a preliminary view
of the research questions, but the overall reliability of these
observations is limited by suspected occurrences of poor data provenance
among some of the data collectors. On the whole, it is interesting that
the final proportions were so close to those of the ``ideal'' man given
the non-ideal circumstances of data collection.

\newpage
\section{APPENDICES}
\label{sec:appendix}

\subsection{Data Provenance}
\label{sec:appendix-data-provenance}

For my contribution to the dataset, measurements were collected as
indicated on the handout. My participants were responsible for
collecting and reporting their own data. A spreadsheet with the
mandatory measurements as well as the handout were supplied by email to
the participants. The compiled class dataset was cleaned with the
function cleanMeasureData. This function corrects for data entry
mistakes that inappropriately affected the number of levels viewed for
each variable: quotations, variations on NA, capitalizations,
mispellings, or abbreviations. In some cases, a quantitative variable
and qualitative variable were switched, or incorrect units were
assigned, these mistakes had to be adjusted on a row-by-row basis. All
units were converted to inches, expressed as ``in.'' Variables for which
sidedness was a factor were averaged, but the individual sides were
retained in order to investigate the third research question on limb
dominance. A variable for ape-index was created by subtracting the
height by the armspan for each respective column. All individuals that
were missing data for height for armspan information were removed. All
duplicate rows were removed to ensure that only unique observations
informed the data analysis.

\newpage
\subsubsection{Data Collection Handout}
\label{sec:appendix-data-handout}

\begin{figure}[!ht]
    \hrule
    \caption{ \textbf{Handout Page 1} }
    \begin{center}
        \scalebox{1.00}{    \includegraphics[trim = 0 0 0 0,clip,width=0.85\textwidth]{pdfs/Kornowske_handout_page1.pdf} }
    \end{center}
    \label{fig:handout-1}
    \hrule
\end{figure}

Handout was assembled with the tools provided by Canva.com

\newpage

\begin{figure}[!ht]
    \hrule
    \caption{ \textbf{Handout Page 2} }
    \begin{center}
        \scalebox{1.00}{    \includegraphics[trim = 0 0 0 0,clip,width=0.85\textwidth]{pdfs/Kornowske_handout_Page2.pdf} }
    \end{center}
    \label{fig:handout-2}
    \hrule
\end{figure}

\newpage

\newpage

\subsection{Data, Data Cleaning, and Analysis}
\label{sec:appendix-setup}

Below is the necessary functions and libraries required to run the code
referenced in this document.

\begin{Shaded}
\begin{Highlighting}[]
\KeywordTok{library}\NormalTok{(devtools);       }\CommentTok{\# required for source\_url}
\KeywordTok{library}\NormalTok{(tidyverse);}
\KeywordTok{library}\NormalTok{(Hmisc);}
\KeywordTok{library}\NormalTok{(kableExtra);}

\NormalTok{path.humanVerseWSU =}\StringTok{ "https://raw.githubusercontent.com/MonteShaffer/humanVerseWSU/"}
\KeywordTok{source\_url}\NormalTok{( }\KeywordTok{paste0}\NormalTok{(path.humanVerseWSU,}\StringTok{"master/misc/functions{-}project{-}measure.R"}\NormalTok{) );}
\end{Highlighting}
\end{Shaded}

\subsubsection{Analysis:  Which biometrics are best correlated with height?}
\label{sec:anrq}

Below is the code to load the data and prepare it for analysis.

\begin{Shaded}
\begin{Highlighting}[]
\NormalTok{path.project =}\StringTok{ "/Users/lindseykornowske/.git/STAT419/project{-}measure/"}\NormalTok{;}
\NormalTok{path.to.secret =}\StringTok{ "/Users/lindseykornowske/Documents/FS Classes HW \& Resources/STAT 419 Multivariate/\_SECRET\_/"}\NormalTok{;}
\NormalTok{measure =}\StringTok{ }\NormalTok{utils}\OperatorTok{::}\KeywordTok{read.csv}\NormalTok{( }\KeywordTok{paste0}\NormalTok{(path.to.secret, }\StringTok{"measure{-}students.txt"}\NormalTok{), }\DataTypeTok{header=}\OtherTok{TRUE}\NormalTok{, }\DataTypeTok{sep=}\StringTok{"|"}\NormalTok{);}
\KeywordTok{source}\NormalTok{(}\DataTypeTok{file =} \StringTok{"https://raw.githubusercontent.com/lindseymaek/STAT419/master/functions/functions{-}project{-}measure.R"}\NormalTok{,}\DataTypeTok{local =}\NormalTok{ F);}
\NormalTok{measure.df =}\StringTok{ }\KeywordTok{prepareMeasureData}\NormalTok{(measure);}

\NormalTok{sum.list =}\StringTok{ }\KeywordTok{list}\NormalTok{(}\KeywordTok{summary}\NormalTok{(measure.df[,}\KeywordTok{c}\NormalTok{(}\DecValTok{3}\OperatorTok{:}\DecValTok{9}\NormalTok{)]),}
\KeywordTok{summary}\NormalTok{(measure.df[,}\KeywordTok{c}\NormalTok{(}\DecValTok{11}\OperatorTok{:}\DecValTok{16}\NormalTok{)]),}
\KeywordTok{summary}\NormalTok{(measure.df[,}\KeywordTok{c}\NormalTok{(}\DecValTok{17}\OperatorTok{:}\DecValTok{21}\NormalTok{)]),}
\KeywordTok{summary}\NormalTok{(measure.df[,}\KeywordTok{c}\NormalTok{(}\DecValTok{22}\OperatorTok{:}\DecValTok{27}\NormalTok{)]),}
\KeywordTok{summary}\NormalTok{(measure.df[,}\KeywordTok{c}\NormalTok{(}\DecValTok{28}\OperatorTok{:}\DecValTok{33}\NormalTok{)]),}
\KeywordTok{summary}\NormalTok{(measure.df[,}\KeywordTok{c}\NormalTok{(}\DecValTok{34}\OperatorTok{:}\DecValTok{39}\NormalTok{)]),}
\KeywordTok{summary}\NormalTok{(measure.df[,}\KeywordTok{c}\NormalTok{(}\DecValTok{40}\OperatorTok{:}\DecValTok{45}\NormalTok{)]),}
\KeywordTok{summary}\NormalTok{(measure.df[,}\KeywordTok{c}\NormalTok{(}\DecValTok{46}\OperatorTok{:}\DecValTok{47}\NormalTok{)]));}

\KeywordTok{kable}\NormalTok{(sum.list, }\DataTypeTok{caption =} \StringTok{"Summary Statistics"}\NormalTok{, }\DataTypeTok{format =} \StringTok{"simple"}\NormalTok{) }\OperatorTok{\%\textgreater{}\%}\StringTok{ }\KeywordTok{kable\_styling}\NormalTok{(}\DataTypeTok{latex\_options =} \StringTok{"scale\_down"}\NormalTok{, }\DataTypeTok{font\_size =} \DecValTok{7}\NormalTok{ );}
\end{Highlighting}
\end{Shaded}

\begin{longtable}[]{@{}lccccccc@{}}
\toprule
& side & writing & eye & eye\_color & swinging & gender &
ethnicity\tabularnewline
\midrule
\endhead
& left : 51 & both : 3 & both : 24 & brown :113 & both : 3 & female :122
& caucasian :182\tabularnewline
& NA : 1 & left : 32 & brown: 0 & blue : 78 & left : 49 & male :126 &
asian : 29\tabularnewline
& right:161 & right:214 & left : 71 & hazel : 31 & right:197 &
non-binary: 1 & african american: 7\tabularnewline
& NA's : 36 & NA & right:150 & green : 16 & NA & NA & hispanic :
6\tabularnewline
& NA & NA & NA's : 4 & blue/green: 4 & NA & NA & korean :
5\tabularnewline
& NA & NA & NA & black : 3 & NA & NA & caucasian/asian :
4\tabularnewline
& NA & NA & NA & (Other) : 4 & NA & NA & (Other) : 16\tabularnewline
\bottomrule
\end{longtable}

\begin{longtable}[]{@{}lccclcl@{}}
\toprule
& age & quality & minutes & units & height.NA &
head.height.NA\tabularnewline
\midrule
\endhead
& Min. : 1.00 & Min. : 5.000 & Min. :-2.209e+09 & in:249 & Min. :10.75 &
Min. : 2.559\tabularnewline
& 1st Qu.:22.00 & 1st Qu.: 8.000 & 1st Qu.: 1.000e+01 & NA & 1st
Qu.:62.99 & 1st Qu.: 8.071\tabularnewline
& Median :27.00 & Median : 9.000 & Median : 1.500e+01 & NA & Median
:66.26 & Median : 8.661\tabularnewline
& Mean :34.55 & Mean : 8.582 & Mean :-9.128e+07 & NA & Mean :65.30 &
Mean : 8.699\tabularnewline
& 3rd Qu.:51.00 & 3rd Qu.:10.000 & 3rd Qu.: 2.000e+01 & NA & 3rd
Qu.:70.00 & 3rd Qu.: 9.250\tabularnewline
& Max. :94.00 & Max. :10.000 & Max. : 4.500e+01 & NA & Max. :75.24 &
Max. :15.000\tabularnewline
& NA & NA & NA's :7 & NA & NA & NA's :9\tabularnewline
\bottomrule
\end{longtable}

\begin{longtable}[]{@{}llllll@{}}
\toprule
& head.circumference.NA & hand.length.left & hand.length.right &
hand.width.left & hand.width.right\tabularnewline
\midrule
\endhead
& Min. : 7.874 & Min. : 1.968 & Min. : 1.968 & Min. : 2.165 & Min. :
2.165\tabularnewline
& 1st Qu.:21.654 & 1st Qu.: 6.693 & 1st Qu.: 6.693 & 1st Qu.: 7.250 &
1st Qu.: 7.395\tabularnewline
& Median :22.250 & Median : 7.125 & Median : 7.087 & Median : 7.874 &
Median : 7.874\tabularnewline
& Mean :22.000 & Mean : 7.705 & Mean : 7.681 & Mean : 7.702 & Mean :
7.720\tabularnewline
& 3rd Qu.:23.000 & 3rd Qu.: 7.677 & 3rd Qu.: 7.677 & 3rd Qu.: 8.465 &
3rd Qu.: 8.473\tabularnewline
& Max. :25.236 & Max. :87.875 & Max. :87.875 & Max. :10.433 & Max.
:10.433\tabularnewline
& NA's :16 & NA's :6 & NA's :2 & NA's :16 & NA's :13\tabularnewline
\bottomrule
\end{longtable}

\begin{longtable}[]{@{}lllllll@{}}
\toprule
& hand.elbow.left & hand.elbow.right & elbow.armpit.left &
elbow.armpit.right & arm.reach.left & arm.reach.right\tabularnewline
\midrule
\endhead
& Min. : 4.429 & Min. : 4.429 & Min. : 2.362 & Min. : 2.362 & Min.
:14.96 & Min. :20.47\tabularnewline
& 1st Qu.:15.750 & 1st Qu.:15.750 & 1st Qu.: 9.004 & 1st Qu.: 9.000 &
1st Qu.:74.33 & 1st Qu.:74.05\tabularnewline
& Median :16.929 & Median :16.929 & Median :10.118 & Median :10.039 &
Median :81.10 & Median :81.10\tabularnewline
& Mean :16.664 & Mean :16.650 & Mean :10.452 & Mean :10.481 & Mean
:74.61 & Mean :74.93\tabularnewline
& 3rd Qu.:17.913 & 3rd Qu.:17.913 & 3rd Qu.:11.500 & 3rd Qu.:11.811 &
3rd Qu.:87.22 & 3rd Qu.:87.62\tabularnewline
& Max. :20.472 & Max. :20.669 & Max. :27.953 & Max. :27.953 & Max.
:96.46 & Max. :96.06\tabularnewline
& NA's :16 & NA's :13 & NA's :15 & NA's :12 & NA's :17 & NA's
:14\tabularnewline
\bottomrule
\end{longtable}

\begin{longtable}[]{@{}lclllll@{}}
\toprule
& arm.span.NA & foot.length.left & foot.length.right &
floor.kneepit.left & floor.kneepit.right & floor.hip.left\tabularnewline
\midrule
\endhead
& Min. : 3.50 & Min. : 2.756 & Min. : 2.756 & Min. : 5.118 & Min. :
5.118 & Min. : 9.646\tabularnewline
& 1st Qu.:62.25 & 1st Qu.: 9.055 & 1st Qu.: 9.055 & 1st Qu.:16.535 & 1st
Qu.:16.500 & 1st Qu.:35.710\tabularnewline
& Median :66.14 & Median : 9.750 & Median : 9.750 & Median :17.750 &
Median :17.795 & Median :37.795\tabularnewline
& Mean :65.28 & Mean : 9.703 & Mean : 9.735 & Mean :17.733 & Mean
:17.781 & Mean :37.266\tabularnewline
& 3rd Qu.:70.47 & 3rd Qu.:10.287 & 3rd Qu.:10.433 & 3rd Qu.:19.000 & 3rd
Qu.:19.000 & 3rd Qu.:39.764\tabularnewline
& Max. :88.19 & Max. :15.000 & Max. :18.504 & Max. :28.425 & Max.
:39.764 & Max. :44.488\tabularnewline
& NA & NA's :15 & NA's :12 & NA's :15 & NA's :12 & NA's
:25\tabularnewline
\bottomrule
\end{longtable}

\begin{longtable}[]{@{}lllllcc@{}}
\toprule
& floor.hip.right & floor.navel.NA & floor.armpit.left &
floor.armpit.right & hand.length & hand.width\tabularnewline
\midrule
\endhead
& Min. : 9.646 & Min. :10.63 & Min. :13.98 & Min. :13.78 & Min. : 1.968
& Min. : 2.165\tabularnewline
& 1st Qu.:35.467 & 1st Qu.:37.40 & 1st Qu.:49.00 & 1st Qu.:48.43 & 1st
Qu.: 6.732 & 1st Qu.: 7.336\tabularnewline
& Median :37.795 & Median :39.76 & Median :51.70 & Median :51.50 &
Median : 7.165 & Median : 7.874\tabularnewline
& Mean :37.228 & Mean :39.87 & Mean :51.46 & Mean :50.76 & Mean : 7.712
& Mean : 7.725\tabularnewline
& 3rd Qu.:39.764 & 3rd Qu.:42.13 & 3rd Qu.:54.93 & 3rd Qu.:54.72 & 3rd
Qu.: 7.675 & 3rd Qu.: 8.465\tabularnewline
& Max. :44.488 & Max. :67.00 & Max. :61.71 & Max. :62.00 & Max. :87.875
& Max. :10.433\tabularnewline
& NA's :22 & NA's :40 & NA's :15 & NA's :14 & NA's :8 & NA's
:19\tabularnewline
\bottomrule
\end{longtable}

\begin{longtable}[]{@{}lccclcc@{}}
\toprule
& floor.armpit & hand.elbow & elbow.armpit & floor.kneepit & floor.hip &
arm.reach\tabularnewline
\midrule
\endhead
& Min. :13.88 & Min. : 4.429 & Min. : 2.362 & Min. : 5.118 & Min. :
9.646 & Min. :20.67\tabularnewline
& 1st Qu.:48.62 & 1st Qu.:15.901 & 1st Qu.: 9.000 & 1st Qu.:16.535 & 1st
Qu.:35.500 & 1st Qu.:75.12\tabularnewline
& Median :51.50 & Median :16.929 & Median :10.150 & Median :17.815 &
Median :37.795 & Median :81.40\tabularnewline
& Mean :51.14 & Mean :16.686 & Mean :10.478 & Mean :17.806 & Mean
:37.273 & Mean :74.89\tabularnewline
& 3rd Qu.:54.72 & 3rd Qu.:17.928 & 3rd Qu.:11.620 & 3rd Qu.:19.094 & 3rd
Qu.:39.764 & 3rd Qu.:87.40\tabularnewline
& Max. :61.61 & Max. :20.571 & Max. :27.953 & Max. :29.134 & Max.
:44.488 & Max. :96.26\tabularnewline
& NA's :20 & NA's :19 & NA's :18 & NA's :18 & NA's :28 & NA's
:20\tabularnewline
\bottomrule
\end{longtable}

\begin{longtable}[]{@{}lcc@{}}
\toprule
& foot.length & ape.index\tabularnewline
\midrule
\endhead
& Min. : 2.756 & Min. :-59.00000\tabularnewline
& 1st Qu.: 9.090 & 1st Qu.: -1.18110\tabularnewline
& Median : 9.803 & Median : 0.30000\tabularnewline
& Mean : 9.732 & Mean : -0.01708\tabularnewline
& 3rd Qu.:10.375 & 3rd Qu.: 1.75000\tabularnewline
& Max. :15.256 & Max. : 61.95000\tabularnewline
& NA's :18 & NA\tabularnewline
\bottomrule
\end{longtable}

Below is the code to generate the summary statistics and save them as a
table.

\begin{Shaded}
\begin{Highlighting}[]
\NormalTok{measure.collapsed =}\StringTok{ }\NormalTok{measure.df[,}\KeywordTok{c}\NormalTok{(}\DecValTok{15}\OperatorTok{:}\DecValTok{17}\NormalTok{,}\DecValTok{28}\NormalTok{,}\DecValTok{35}\NormalTok{, }\DecValTok{38}\OperatorTok{:}\DecValTok{46}\NormalTok{)]}
\NormalTok{names =}\StringTok{ }\KeywordTok{colnames}\NormalTok{(measure.collapsed);}
\NormalTok{summary =}\StringTok{ }\KeywordTok{summarizeMeanSD}\NormalTok{(measure.collapsed);}

\NormalTok{mean =}\StringTok{ }\KeywordTok{t}\NormalTok{(summary[}\DecValTok{1}\NormalTok{,]);}
\NormalTok{sd =}\StringTok{ }\KeywordTok{t}\NormalTok{(summary[}\DecValTok{2}\NormalTok{,]);}
\NormalTok{measure.scale =}\StringTok{ }\NormalTok{measure.collapsed}\OperatorTok{/}\NormalTok{measure.collapsed[,}\DecValTok{1}\NormalTok{];}

\NormalTok{my.corr =}\StringTok{ }\KeywordTok{rcorr}\NormalTok{( }\KeywordTok{as.matrix}\NormalTok{(measure.scale), }\DataTypeTok{type=}\StringTok{"pearson"}\NormalTok{);}

\CommentTok{\#str(my.corr);}

\NormalTok{my.corr.r =}\StringTok{ }\NormalTok{my.corr}\OperatorTok{$}\NormalTok{r;}
\NormalTok{my.corr.pval =}\StringTok{ }\NormalTok{my.corr}\OperatorTok{$}\NormalTok{P;}

\NormalTok{my.corr.r}\FloatTok{.2}\NormalTok{ =}\StringTok{ }\KeywordTok{round}\NormalTok{(my.corr.r,}\DecValTok{2}\NormalTok{);}
\NormalTok{my.corr.p}\FloatTok{.3}\NormalTok{ =}\StringTok{ }\KeywordTok{as.numeric}\NormalTok{( }\KeywordTok{round}\NormalTok{(my.corr.pval,}\DecValTok{3}\NormalTok{) ); }\CommentTok{\# flatten}

\NormalTok{cuts =}\StringTok{ }\KeywordTok{c}\NormalTok{(}\FloatTok{0.10}\NormalTok{, }\FloatTok{0.05}\NormalTok{, }\FloatTok{0.01}\NormalTok{, }\FloatTok{0.001}\NormalTok{);}
\NormalTok{symb =}\StringTok{ }\KeywordTok{c}\NormalTok{(}\StringTok{"+"}\NormalTok{, }\StringTok{"*"}\NormalTok{, }\StringTok{"**"}\NormalTok{, }\StringTok{"***"}\NormalTok{);}

\NormalTok{my.corr.p.}\FloatTok{3.}\NormalTok{symb =}\StringTok{ ""}\NormalTok{;}
\NormalTok{my.corr.p.}\FloatTok{3.}\NormalTok{symb[}\KeywordTok{is.na}\NormalTok{(my.corr.p}\FloatTok{.3}\NormalTok{)] =}\StringTok{ ""}\NormalTok{;}
\NormalTok{my.corr.p.}\FloatTok{3.}\NormalTok{symb[my.corr.p}\FloatTok{.3} \OperatorTok{\textless{}=}\StringTok{ }\FloatTok{0.10}\NormalTok{] =}\StringTok{ "+"}\NormalTok{;}
\NormalTok{my.corr.p.}\FloatTok{3.}\NormalTok{symb[my.corr.p}\FloatTok{.3} \OperatorTok{\textless{}=}\StringTok{ }\FloatTok{0.05}\NormalTok{] =}\StringTok{ "*"}\NormalTok{;}
\NormalTok{my.corr.p.}\FloatTok{3.}\NormalTok{symb[my.corr.p}\FloatTok{.3} \OperatorTok{\textless{}=}\StringTok{ }\FloatTok{0.01}\NormalTok{] =}\StringTok{ "**"}\NormalTok{;}
\NormalTok{my.corr.p.}\FloatTok{3.}\NormalTok{symb[my.corr.p}\FloatTok{.3} \OperatorTok{\textless{}=}\StringTok{ }\FloatTok{0.001}\NormalTok{] =}\StringTok{ "***"}\NormalTok{;}

\NormalTok{include.diag =}\StringTok{ }\OtherTok{FALSE}\NormalTok{;  }\CommentTok{\# the 1\textquotesingle{}s on the diagonal are not included}
\CommentTok{\# this is a lower triangular form ...}

\NormalTok{char.matrix =}\StringTok{ }\KeywordTok{as.character}\NormalTok{(my.corr.r}\FloatTok{.2}\NormalTok{);  }


\NormalTok{my.matrix =}\StringTok{ }\KeywordTok{matrix}\NormalTok{( }
                \KeywordTok{paste0}\NormalTok{(char.matrix, my.corr.p.}\FloatTok{3.}\NormalTok{symb),}
                \DataTypeTok{nrow=}\KeywordTok{ncol}\NormalTok{(measure.collapsed));}
\NormalTok{my.matrix =}\StringTok{ }\KeywordTok{as.data.frame}\NormalTok{(my.matrix);}
\KeywordTok{colnames}\NormalTok{(my.matrix) =}\StringTok{ }\NormalTok{names;}
\KeywordTok{rownames}\NormalTok{(my.matrix) =}\StringTok{ }\NormalTok{names;}
\NormalTok{my.matrix}\OperatorTok{$}\NormalTok{mean =}\StringTok{ }\NormalTok{mean;}
\NormalTok{my.matrix}\OperatorTok{$}\NormalTok{sd =}\StringTok{ }\NormalTok{sd[}\DecValTok{1}\NormalTok{,];}
\NormalTok{my.matrix =}\StringTok{ }\NormalTok{my.matrix[,}\KeywordTok{c}\NormalTok{(}\DecValTok{15}\NormalTok{,}\DecValTok{16}\NormalTok{,}\DecValTok{1}\OperatorTok{:}\DecValTok{14}\NormalTok{)];}

\NormalTok{my.matrix}\FloatTok{.1}\NormalTok{=my.matrix[,}\KeywordTok{c}\NormalTok{(}\DecValTok{1}\OperatorTok{:}\DecValTok{4}\NormalTok{)];}
\NormalTok{my.matrix}\FloatTok{.2}\NormalTok{=my.matrix[,}\KeywordTok{c}\NormalTok{(}\DecValTok{6}\OperatorTok{:}\DecValTok{8}\NormalTok{)];}
\NormalTok{my.matrix}\FloatTok{.3}\NormalTok{=my.matrix[,}\KeywordTok{c}\NormalTok{(}\DecValTok{9}\OperatorTok{:}\DecValTok{12}\NormalTok{)];}
\NormalTok{my.matrix}\FloatTok{.4}\NormalTok{=my.matrix[,}\KeywordTok{c}\NormalTok{(}\DecValTok{13}\OperatorTok{:}\DecValTok{16}\NormalTok{)];}

\KeywordTok{kable}\NormalTok{(my.matrix}\FloatTok{.1}\NormalTok{,}
      \DataTypeTok{format =} \StringTok{"latex"}\NormalTok{, }\DataTypeTok{booktabs =} \OtherTok{TRUE}\NormalTok{) }\OperatorTok{\%\textgreater{}\%}\StringTok{ }\KeywordTok{kable\_styling}\NormalTok{(}\DataTypeTok{latex\_options =} \StringTok{"scale\_down"}\NormalTok{, }\DataTypeTok{font\_size =} \DecValTok{7}\NormalTok{ )}
\end{Highlighting}
\end{Shaded}

\begin{table}[H]
\centering\begingroup\fontsize{7}{9}\selectfont

\resizebox{\linewidth}{!}{
\begin{tabular}{lrrll}
\toprule
  & mean & sd & height.NA & head.height.NA\\
\midrule
height.NA & 65.3 & 8.02 & 1 & NaN\\
head.height.NA & 8.7 & 8.02 & NaN & 1\\
head.circumference.NA & 22.0 & 8.02 & NaN & 0.93***\\
arm.span.NA & 65.3 & 8.02 & NaN & 0.86***\\
floor.navel.NA & 39.9 & 8.02 & NaN & 0.9***\\
\addlinespace
hand.length & 7.7 & 8.02 & NaN & 0.33***\\
hand.width & 7.7 & 8.02 & NaN & 0.89***\\
floor.armpit & 51.1 & 8.02 & NaN & 0.91***\\
hand.elbow & 16.7 & 8.02 & NaN & 0.91***\\
elbow.armpit & 10.5 & 8.02 & NaN & 0.78***\\
\addlinespace
floor.kneepit & 17.8 & 8.02 & NaN & 0.9***\\
floor.hip & 37.3 & 8.02 & NaN & 0.91***\\
arm.reach & 74.9 & 8.02 & NaN & 0.8***\\
foot.length & 9.7 & 8.02 & NaN & 0.91***\\
\bottomrule
\end{tabular}}
\endgroup{}
\end{table}

\begin{Shaded}
\begin{Highlighting}[]
\KeywordTok{kable}\NormalTok{(my.matrix}\FloatTok{.2}\NormalTok{,}
      \DataTypeTok{format =} \StringTok{"latex"}\NormalTok{, }\DataTypeTok{booktabs =} \OtherTok{TRUE}\NormalTok{) }\OperatorTok{\%\textgreater{}\%}\StringTok{ }\KeywordTok{kable\_styling}\NormalTok{(}\DataTypeTok{latex\_options =} \StringTok{"scale\_down"}\NormalTok{, }\DataTypeTok{font\_size =} \DecValTok{7}\NormalTok{ )}
\end{Highlighting}
\end{Shaded}

\begin{table}[H]
\centering\begingroup\fontsize{7}{9}\selectfont

\resizebox{\linewidth}{!}{
\begin{tabular}{llll}
\toprule
  & arm.span.NA & floor.navel.NA & hand.length\\
\midrule
height.NA & NaN & NaN & NaN\\
head.height.NA & 0.86*** & 0.9*** & 0.33***\\
head.circumference.NA & 0.89*** & 0.92*** & 0.36***\\
arm.span.NA & 1 & 0.93*** & 0.38***\\
floor.navel.NA & 0.93*** & 1 & 0.38***\\
\addlinespace
hand.length & 0.38*** & 0.38*** & 1\\
hand.width & 0.93*** & 0.94*** & 0.36***\\
floor.armpit & 0.95*** & 0.96*** & 0.39***\\
hand.elbow & 0.95*** & 0.96*** & 0.38***\\
elbow.armpit & 0.83*** & 0.85*** & 0.34***\\
\addlinespace
floor.kneepit & 0.92*** & 0.95*** & 0.38***\\
floor.hip & 0.95*** & 0.98*** & 0.39***\\
arm.reach & 0.85*** & 0.81*** & 0.36***\\
foot.length & 0.95*** & 0.96*** & 0.38***\\
\bottomrule
\end{tabular}}
\endgroup{}
\end{table}

\begin{Shaded}
\begin{Highlighting}[]
\KeywordTok{kable}\NormalTok{(my.matrix}\FloatTok{.3}\NormalTok{,}
      \DataTypeTok{format =} \StringTok{"latex"}\NormalTok{, }\DataTypeTok{booktabs =} \OtherTok{TRUE}\NormalTok{) }\OperatorTok{\%\textgreater{}\%}\StringTok{ }\KeywordTok{kable\_styling}\NormalTok{(}\DataTypeTok{latex\_options =} \StringTok{"scale\_down"}\NormalTok{, }\DataTypeTok{font\_size =} \DecValTok{7}\NormalTok{ )}
\end{Highlighting}
\end{Shaded}

\begin{table}[H]
\centering\begingroup\fontsize{7}{9}\selectfont

\resizebox{\linewidth}{!}{
\begin{tabular}{lllll}
\toprule
  & hand.width & floor.armpit & hand.elbow & elbow.armpit\\
\midrule
height.NA & NaN & NaN & NaN & NaN\\
head.height.NA & 0.89*** & 0.91*** & 0.91*** & 0.78***\\
head.circumference.NA & 0.91*** & 0.94*** & 0.93*** & 0.78***\\
arm.span.NA & 0.93*** & 0.95*** & 0.95*** & 0.83***\\
floor.navel.NA & 0.94*** & 0.96*** & 0.96*** & 0.85***\\
\addlinespace
hand.length & 0.36*** & 0.39*** & 0.38*** & 0.34***\\
hand.width & 1 & 0.94*** & 0.94*** & 0.8***\\
floor.armpit & 0.94*** & 1 & 0.97*** & 0.86***\\
hand.elbow & 0.94*** & 0.97*** & 1 & 0.87***\\
elbow.armpit & 0.8*** & 0.86*** & 0.87*** & 1\\
\addlinespace
floor.kneepit & 0.9*** & 0.96*** & 0.95*** & 0.85***\\
floor.hip & 0.94*** & 0.98*** & 0.96*** & 0.85***\\
arm.reach & 0.88*** & 0.88*** & 0.83*** & 0.7***\\
foot.length & 0.94*** & 0.97*** & 0.97*** & 0.85***\\
\bottomrule
\end{tabular}}
\endgroup{}
\end{table}

\begin{Shaded}
\begin{Highlighting}[]
\KeywordTok{kable}\NormalTok{(my.matrix}\FloatTok{.4}\NormalTok{,}
      \DataTypeTok{format =} \StringTok{"latex"}\NormalTok{, }\DataTypeTok{booktabs =} \OtherTok{TRUE}\NormalTok{) }\OperatorTok{\%\textgreater{}\%}\StringTok{ }\KeywordTok{kable\_styling}\NormalTok{(}\DataTypeTok{latex\_options =} \StringTok{"scale\_down"}\NormalTok{, }\DataTypeTok{font\_size =} \DecValTok{7}\NormalTok{ )}
\end{Highlighting}
\end{Shaded}

\begin{table}[H]
\centering\begingroup\fontsize{7}{9}\selectfont

\resizebox{\linewidth}{!}{
\begin{tabular}{lllll}
\toprule
  & floor.kneepit & floor.hip & arm.reach & foot.length\\
\midrule
height.NA & NaN & NaN & NaN & NaN\\
head.height.NA & 0.9*** & 0.91*** & 0.8*** & 0.91***\\
head.circumference.NA & 0.93*** & 0.94*** & 0.83*** & 0.94***\\
arm.span.NA & 0.92*** & 0.95*** & 0.85*** & 0.95***\\
floor.navel.NA & 0.95*** & 0.98*** & 0.81*** & 0.96***\\
\addlinespace
hand.length & 0.38*** & 0.39*** & 0.36*** & 0.38***\\
hand.width & 0.9*** & 0.94*** & 0.88*** & 0.94***\\
floor.armpit & 0.96*** & 0.98*** & 0.88*** & 0.97***\\
hand.elbow & 0.95*** & 0.96*** & 0.83*** & 0.97***\\
elbow.armpit & 0.85*** & 0.85*** & 0.7*** & 0.85***\\
\addlinespace
floor.kneepit & 1 & 0.95*** & 0.83*** & 0.96***\\
floor.hip & 0.95*** & 1 & 0.88*** & 0.97***\\
arm.reach & 0.83*** & 0.88*** & 1 & 0.86***\\
foot.length & 0.96*** & 0.97*** & 0.86*** & 1\\
\bottomrule
\end{tabular}}
\endgroup{}
\end{table}
\newpage

\subsubsection{Analysis: How do famous climbers compare to the FALL 2020 STAT 419 Measure dataset for ape index?}
\label{sec:anrq2}

\begin{Shaded}
\begin{Highlighting}[]
\CommentTok{\#read in climber data}
\NormalTok{measure.climbers =}\StringTok{ }\NormalTok{utils}\OperatorTok{::}\KeywordTok{read.csv}\NormalTok{( }\KeywordTok{paste0}\NormalTok{(path.to.secret, }\StringTok{"measure\_climbers.csv"}\NormalTok{), }\DataTypeTok{header=}\OtherTok{TRUE}\NormalTok{, }\DataTypeTok{quote=}\StringTok{""}\NormalTok{, }\DataTypeTok{sep=}\StringTok{","}\NormalTok{);}
\NormalTok{climbers =}\StringTok{ }\NormalTok{measure.climbers}\OperatorTok{$}\NormalTok{Climber;}
\KeywordTok{rownames}\NormalTok{(measure.climbers) =}\StringTok{ }\NormalTok{climbers;}
\NormalTok{measure.climbers =}\StringTok{ }\NormalTok{measure.climbers[,}\KeywordTok{c}\NormalTok{(}\DecValTok{2}\OperatorTok{:}\DecValTok{4}\NormalTok{)];}
\KeywordTok{colnames}\NormalTok{(measure.climbers) =}\StringTok{ }\KeywordTok{c}\NormalTok{(}\StringTok{"height.NA"}\NormalTok{, }\StringTok{"arm.span.NA"}\NormalTok{, }\StringTok{"ape.index"}\NormalTok{);}

\KeywordTok{summarizeMeanSD}\NormalTok{(measure.climbers);}
\end{Highlighting}
\end{Shaded}

\begin{tabular}{l|r|r|r}
\hline
  & height.NA & arm.span.NA & ape.index\\
\hline
mean & 67.50 & 69.80 & 2.30\\
\hline
standard deviation & 4.36 & 5.31 & 1.75\\
\hline
\end{tabular}

\begin{Shaded}
\begin{Highlighting}[]
\NormalTok{measure.rq2 =}\StringTok{ }\KeywordTok{as.data.frame}\NormalTok{(}\KeywordTok{cbind}\NormalTok{(measure.df}\OperatorTok{$}\NormalTok{height.NA, measure.df}\OperatorTok{$}\NormalTok{arm.span.NA, measure.df}\OperatorTok{$}\NormalTok{ape.index));}
\KeywordTok{rownames}\NormalTok{(measure.rq2) =}\StringTok{ }\KeywordTok{rownames}\NormalTok{(measure.df);}
\KeywordTok{colnames}\NormalTok{(measure.rq2) =}\StringTok{ }\KeywordTok{c}\NormalTok{(}\StringTok{"height.NA"}\NormalTok{, }\StringTok{"arm.span.NA"}\NormalTok{, }\StringTok{"ape.index"}\NormalTok{);}
  
\NormalTok{measure.rq2 =}\StringTok{ }\KeywordTok{as.data.frame}\NormalTok{(}\KeywordTok{rbind}\NormalTok{(measure.climbers,measure.rq2));}
\NormalTok{measure.rq2 =}\StringTok{ }\KeywordTok{na.omit}\NormalTok{(measure.rq2);}

\NormalTok{m.kmeans =}\StringTok{ }\KeywordTok{kmeans}\NormalTok{(measure.rq2, }\DecValTok{5}\NormalTok{);}

\NormalTok{m.kmeans}\OperatorTok{$}\NormalTok{centers;}
\end{Highlighting}
\end{Shaded}

\begin{verbatim}
##   height.NA arm.span.NA   ape.index
## 1  67.89193    67.38850  -0.5036246
## 2  71.15596    73.18522   2.0289076
## 3  46.04543    31.70260 -14.3428225
## 4  63.37854    82.38575  19.0072047
## 5  61.88488    61.48858  -0.3963022
\end{verbatim}

\begin{Shaded}
\begin{Highlighting}[]
\NormalTok{membership =}\StringTok{ }\KeywordTok{as.data.frame}\NormalTok{( }\KeywordTok{matrix}\NormalTok{( m.kmeans}\OperatorTok{$}\NormalTok{cluster, }\DataTypeTok{ncol=}\DecValTok{1}\NormalTok{)) ;}
    
\KeywordTok{rownames}\NormalTok{(membership) =}\StringTok{ }\KeywordTok{rownames}\NormalTok{(measure.rq2);}


\CommentTok{\#print( table(membership) ) ; }
\end{Highlighting}
\end{Shaded}

\newpage
\subsubsection{Analysis:Are dominant limbs of consistent greater or lesser relative size?}
\label{sec:anrq3}

Below is the code used to assess the number of occurances for which
dominant hand was the larger or smaller of the hand sizes.

\begin{Shaded}
\begin{Highlighting}[]
\NormalTok{measure.rl =}\StringTok{ }\NormalTok{measure.df[,}\KeywordTok{c}\NormalTok{(}\DecValTok{4}\NormalTok{,}\DecValTok{7}\NormalTok{,}\DecValTok{18}\OperatorTok{:}\DecValTok{21}\NormalTok{,}\DecValTok{22}\OperatorTok{:}\DecValTok{25}\NormalTok{)]; }\CommentTok{\#separate variables of interest}

\CommentTok{\#calculate arm length}
\NormalTok{measure.rl}\OperatorTok{$}\NormalTok{left.arm =}\StringTok{ }\NormalTok{measure.rl}\OperatorTok{$}\NormalTok{hand.elbow.left}\OperatorTok{+}\NormalTok{measure.rl}\OperatorTok{$}\NormalTok{elbow.armpit.left;}
\NormalTok{measure.rl}\OperatorTok{$}\NormalTok{right.arm =}\StringTok{ }\NormalTok{measure.rl}\OperatorTok{$}\NormalTok{hand.elbow.right }\OperatorTok{+}\StringTok{ }\NormalTok{measure.rl}\OperatorTok{$}\NormalTok{elbow.armpit.right;}

\NormalTok{measure.rl =}\StringTok{ }\NormalTok{measure.rl[,}\KeywordTok{c}\NormalTok{(}\DecValTok{1}\OperatorTok{:}\DecValTok{6}\NormalTok{,}\DecValTok{11}\NormalTok{,}\DecValTok{12}\NormalTok{)];}

\NormalTok{measure.rl}\OperatorTok{$}\NormalTok{RminusL.arm.diff =}\StringTok{ }\KeywordTok{round}\NormalTok{((measure.rl}\OperatorTok{$}\NormalTok{right.arm }\OperatorTok{{-}}\StringTok{ }\NormalTok{measure.rl}\OperatorTok{$}\NormalTok{left.arm), }\DecValTok{2}\NormalTok{); }\CommentTok{\#positive value if right arm is bigger}
\NormalTok{measure.rl}\OperatorTok{$}\NormalTok{RminusL.hand.diff =}\StringTok{ }\KeywordTok{round}\NormalTok{((measure.rl}\OperatorTok{$}\NormalTok{hand.length.right }\OperatorTok{{-}}\StringTok{ }\NormalTok{measure.rl}\OperatorTok{$}\NormalTok{hand.length.left),}\DecValTok{2}\NormalTok{);}

\NormalTok{measure.rl.diff =}\StringTok{ }\NormalTok{measure.rl[,}\KeywordTok{c}\NormalTok{(}\DecValTok{1}\NormalTok{,}\DecValTok{2}\NormalTok{,}\DecValTok{9}\NormalTok{,}\DecValTok{10}\NormalTok{)];}
\NormalTok{measure.rl.diff =}\StringTok{ }\KeywordTok{na.omit}\NormalTok{(measure.rl.diff);}

\NormalTok{  same =}\StringTok{ }\DecValTok{0}\NormalTok{;}
\NormalTok{  different =}\StringTok{ }\DecValTok{0}\NormalTok{;}
\NormalTok{ndiffs =}\StringTok{ }\KeywordTok{length}\NormalTok{(measure.rl.diff}\OperatorTok{$}\NormalTok{writing);}
\ControlFlowTok{for}\NormalTok{ (i }\ControlFlowTok{in} \DecValTok{1}\OperatorTok{:}\NormalTok{ndiffs) \{}
\NormalTok{  names =}\StringTok{ }\KeywordTok{c}\NormalTok{(}\StringTok{"Same"}\NormalTok{, }\StringTok{"Different"}\NormalTok{)}
\NormalTok{  count.table.hand =}\StringTok{ }\KeywordTok{matrix}\NormalTok{(}\DataTypeTok{ncol =}\DecValTok{2}\NormalTok{);}
  \ControlFlowTok{if}\NormalTok{(measure.rl.diff}\OperatorTok{$}\NormalTok{writing[i] }\OperatorTok{==}\StringTok{ "right"}\NormalTok{)\{}
    \ControlFlowTok{if}\NormalTok{(measure.rl.diff}\OperatorTok{$}\NormalTok{RminusL.hand.diff[i] }\OperatorTok{\textgreater{}}\StringTok{ }\DecValTok{0}\NormalTok{)\{}
\NormalTok{      same =}\StringTok{ }\NormalTok{same }\OperatorTok{+}\DecValTok{1}\NormalTok{;}
\NormalTok{    \} }\ControlFlowTok{else} \ControlFlowTok{if}\NormalTok{(measure.rl.diff}\OperatorTok{$}\NormalTok{RminusL.hand.diff[i] }\OperatorTok{\textless{}}\StringTok{ }\DecValTok{0}\NormalTok{)\{}
\NormalTok{      different =}\StringTok{ }\NormalTok{different }\OperatorTok{+}\DecValTok{1}\NormalTok{;}
\NormalTok{    \}}
\NormalTok{  \} }\ControlFlowTok{else} \ControlFlowTok{if}\NormalTok{(measure.rl.diff}\OperatorTok{$}\NormalTok{writing[i] }\OperatorTok{==}\StringTok{ "left"}\NormalTok{)\{}
    \ControlFlowTok{if}\NormalTok{(measure.rl.diff}\OperatorTok{$}\NormalTok{RminusL.hand.diff[i] }\OperatorTok{\textless{}}\StringTok{ }\DecValTok{0}\NormalTok{)\{}
\NormalTok{      same =}\StringTok{ }\NormalTok{same }\OperatorTok{+}\DecValTok{1}\NormalTok{;}
\NormalTok{    \} }\ControlFlowTok{else} \ControlFlowTok{if}\NormalTok{(measure.rl.diff}\OperatorTok{$}\NormalTok{RminusL.hand.diff[i] }\OperatorTok{\textgreater{}}\StringTok{ }\DecValTok{0}\NormalTok{)\{}
\NormalTok{      different =}\StringTok{ }\NormalTok{different }\OperatorTok{+}\DecValTok{1}\NormalTok{;}
\NormalTok{    \}}
\NormalTok{  \}}
\NormalTok{  count.table.hand[}\DecValTok{1}\NormalTok{,}\DecValTok{1}\NormalTok{] =}\StringTok{ }\NormalTok{same;}
\NormalTok{  count.table.hand[}\DecValTok{1}\NormalTok{,}\DecValTok{2}\NormalTok{] =}\StringTok{ }\NormalTok{different;}
\NormalTok{\}}

\KeywordTok{colnames}\NormalTok{(count.table.hand) =}\StringTok{ }\KeywordTok{c}\NormalTok{(}\StringTok{"Dominant Hand Larger"}\NormalTok{, }\StringTok{"Dominant Hand Smaller"}\NormalTok{);}
\KeywordTok{rownames}\NormalTok{(count.table.hand) =}\StringTok{ }\KeywordTok{c}\NormalTok{(}\StringTok{"Count"}\NormalTok{);}
\NormalTok{count.table.hand;}
\end{Highlighting}
\end{Shaded}

\begin{verbatim}
##       Dominant Hand Larger Dominant Hand Smaller
## Count                   48                    52
\end{verbatim}

\begin{Shaded}
\begin{Highlighting}[]
\NormalTok{  same =}\StringTok{ }\DecValTok{0}\NormalTok{;}
\NormalTok{  different =}\StringTok{ }\DecValTok{0}\NormalTok{;}
\NormalTok{ndiffs =}\StringTok{ }\KeywordTok{length}\NormalTok{(measure.rl.diff}\OperatorTok{$}\NormalTok{swinging);}
\ControlFlowTok{for}\NormalTok{ (i }\ControlFlowTok{in} \DecValTok{1}\OperatorTok{:}\NormalTok{ndiffs) \{}
\NormalTok{  names =}\StringTok{ }\KeywordTok{c}\NormalTok{(}\StringTok{"Same"}\NormalTok{, }\StringTok{"Different"}\NormalTok{)}
\NormalTok{  count.table.arm =}\StringTok{ }\KeywordTok{matrix}\NormalTok{(}\DataTypeTok{ncol =}\DecValTok{2}\NormalTok{);}
  \ControlFlowTok{if}\NormalTok{(measure.rl.diff}\OperatorTok{$}\NormalTok{swinging[i] }\OperatorTok{==}\StringTok{ "right"}\NormalTok{)\{}
    \ControlFlowTok{if}\NormalTok{(measure.rl.diff}\OperatorTok{$}\NormalTok{RminusL.arm.diff[i] }\OperatorTok{\textgreater{}}\StringTok{ }\DecValTok{0}\NormalTok{)\{}
\NormalTok{      same =}\StringTok{ }\NormalTok{same }\OperatorTok{+}\DecValTok{1}\NormalTok{;}
\NormalTok{    \} }\ControlFlowTok{else} \ControlFlowTok{if}\NormalTok{(measure.rl.diff}\OperatorTok{$}\NormalTok{RminusL.arm.diff[i] }\OperatorTok{\textless{}}\StringTok{ }\DecValTok{0}\NormalTok{)\{}
\NormalTok{      different =}\StringTok{ }\NormalTok{different }\OperatorTok{+}\DecValTok{1}\NormalTok{;}
\NormalTok{    \}}
\NormalTok{  \} }\ControlFlowTok{else} \ControlFlowTok{if}\NormalTok{(measure.rl.diff}\OperatorTok{$}\NormalTok{swinging[i] }\OperatorTok{==}\StringTok{ "left"}\NormalTok{)\{}
    \ControlFlowTok{if}\NormalTok{(measure.rl.diff}\OperatorTok{$}\NormalTok{RminusL.arm.diff[i] }\OperatorTok{\textless{}}\StringTok{ }\DecValTok{0}\NormalTok{)\{}
\NormalTok{      same =}\StringTok{ }\NormalTok{same }\OperatorTok{+}\DecValTok{1}\NormalTok{;}
\NormalTok{    \} }\ControlFlowTok{else} \ControlFlowTok{if}\NormalTok{(measure.rl.diff}\OperatorTok{$}\NormalTok{RminusL.arm.diff[i] }\OperatorTok{\textgreater{}}\StringTok{ }\DecValTok{0}\NormalTok{)\{}
\NormalTok{      different =}\StringTok{ }\NormalTok{different }\OperatorTok{+}\DecValTok{1}\NormalTok{;}
\NormalTok{    \}}
\NormalTok{  \}}
\NormalTok{  count.table.arm[}\DecValTok{1}\NormalTok{,}\DecValTok{1}\NormalTok{] =}\StringTok{ }\NormalTok{same;}
\NormalTok{  count.table.arm[}\DecValTok{1}\NormalTok{,}\DecValTok{2}\NormalTok{] =}\StringTok{ }\NormalTok{different;}
\NormalTok{\}}

\KeywordTok{colnames}\NormalTok{(count.table.arm) =}\StringTok{ }\KeywordTok{c}\NormalTok{(}\StringTok{"Dominant Arm Larger"}\NormalTok{, }\StringTok{"Dominant Arm Smaller"}\NormalTok{);}
\KeywordTok{rownames}\NormalTok{(count.table.arm) =}\StringTok{ }\KeywordTok{c}\NormalTok{(}\StringTok{"Count"}\NormalTok{);}
\NormalTok{count.table.arm;}
\end{Highlighting}
\end{Shaded}

\begin{verbatim}
##       Dominant Arm Larger Dominant Arm Smaller
## Count                  74                   78
\end{verbatim}

\begin{Shaded}
\begin{Highlighting}[]
\CommentTok{\#\# A different approach I thought about for this is below,}
\CommentTok{\#\#ultimately I was not certain that it was appropriate to calculate correlations this way}

\CommentTok{\#ndiffs = nrow(measure.rl.diff);}
\CommentTok{\#  for (i in 1:ndiffs) \{}
 \CommentTok{\#   if (measure.rl.diff$writing[i] == "right")\{}
  \CommentTok{\#    measure.rl.diff$writingNum[i] = 1;}
   \CommentTok{\# \} else if (measure.rl.diff$writing[i] == "left") \{}
    \CommentTok{\#  measure.rl.diff$writingNum[i] = c({-}1);}
 \CommentTok{\#   \}}
  \CommentTok{\#  measure.rl.diff;}
  \CommentTok{\#\}}
  
\CommentTok{\#    for (i in 1:ndiffs) \{}
\CommentTok{\#    if (measure.rl.diff$swinging[i] == "right")\{}
\CommentTok{\#      measure.rl.diff$swingingNum[i] = 1;}
\CommentTok{\#    \} else if (measure.rl.diff$swinging[i] == "left") \{}
\CommentTok{\#      measure.rl.diff$swingingNum[i] = c({-}1);}
\CommentTok{\#    \}}
\CommentTok{\#    measure.rl.diff;}
\CommentTok{\#  \}}
  
\CommentTok{\#  measure.rl.diff = measure.rl.diff[,{-}c(1,2)];}
  
\CommentTok{\#  my.corr = rcorr( as.matrix(measure.rl.diff), type="pearson");}

\CommentTok{\#my.corr.r = my.corr$r;}
\CommentTok{\#my.corr.pval = my.corr$P;}

\CommentTok{\#my.corr.r;}
\CommentTok{\#  my.corr.r.2 = round(my.corr.r,2);}
\CommentTok{\#my.corr.p.3 = as.numeric( round(my.corr.pval,3) ); \# flatten}
\CommentTok{\#cuts = c(0.10, 0.05, 0.01, 0.001);}
\CommentTok{\#symb = c("+", "*", "**", "***");}
\CommentTok{\#my.corr.p.3.symb = "";}
\CommentTok{\#my.corr.p.3.symb[is.na(my.corr.p.3)] = "";}
\CommentTok{\#my.corr.p.3.symb[my.corr.p.3 \textless{}= 0.10] = "+";}
\CommentTok{\#my.corr.p.3.symb[my.corr.p.3 \textless{}= 0.05] = "*";}
\CommentTok{\#my.corr.p.3.symb[my.corr.p.3 \textless{}= 0.01] = "**";}
\CommentTok{\#my.corr.p.3.symb[my.corr.p.3 \textless{}= 0.001] = "***";}
\CommentTok{\#include.diag = FALSE;  \# the 1\textquotesingle{}s on the diagonal are not included}
\CommentTok{\# this is a lower triangular form ...}

\CommentTok{\#char.matrix = as.character(my.corr.r.2);  }

\CommentTok{\#names = as.list(row.names(my.corr.r));}
\CommentTok{\#my.matrix = matrix( }
\CommentTok{\#                paste0(char.matrix, my.corr.p.3.symb),}
\CommentTok{\#                nrow=ncol(measure.rl.diff),ncol = 4,);}
\CommentTok{\#as.table(my.matrix);}
\end{Highlighting}
\end{Shaded}





%% appendices go here!


\newpage
\theendnotes

%%%%%%%%%%%%%%%%%%%%%%%%%%%%%%%%%%%  biblio %%%%%%%%
\newpage
\begin{auxmulticols}{2}
\singlespacing 
\bibliography{./../biblio/STAT419.bib}

%%%%%%%%%%%%%%%%%%%%%%%%%%%%%%%%%%%  biblio %%%%%%%%
\end{auxmulticols}

\newpage
{
\hypersetup{linkcolor=black}
\setcounter{tocdepth}{3}
\tableofcontents
}



\end{document}